\documentclass[12pt,a4paper]{article}
\usepackage[utf8]{inputenc}
\usepackage[spanish]{babel}
\usepackage{amsmath}
\usepackage{amsfonts}
\usepackage{amssymb}
\usepackage{graphicx}
\usepackage{setspace}
\usepackage[left=2cm,right=2cm,top=2cm,bottom=2cm]{geometry}

\date{\today}

\begin{document}
 	\begin{titlepage}
	\begin{center}
	{\huge \textbf{Universidad Veracruzana}}\\
	\vspace{2cm}  
	{\Large {Vision y Alcance Del Proyecto}}\\
	\vspace{5mm}	
	{\Large {Spotify-Data}}\\
	\begin{figure}[h]
		\centering
		\includegraphics[scale=0.10]{uvlogo}
	\end{figure}
	{\Large {Ingenieria de software}}\\
    \vspace{2cm}
	{\Large {Arenas Cortes Flor Denisse}}\\
	\vspace{5mm}	
	{\Large {Garcia Sosa Oswaldo }}\\
	\vspace{5mm}	
	{\Large {Hernandez Morales Gustavo Adolfo }}\\
	\vspace{5mm}	
	{\Large {Martinez Espinosa Gerardo Ivan}}\\
	\vspace{5mm}	
	{\Large {Medel Ayohua Victor Ivan}}\\
	\vspace{3mm}	
	{\Large {Ramos Garcia Alberto}}\\
	\vspace{2cm}	
    \rule{8cm}{0.5mm} \\ \Large Vo.bo\\ 
	\end{center}
\end{titlepage}
\newpage
\section{Generalidades}
\subsection{Introducción}
En el presente documento se identifican los aspectos correspondientes a la visión y alcance de este proyecto.\\
\\ El proyecto Spotify-Data es un proyecto donde se requiere poner a prueba la manipulación
de una base de datos con al menos 1 millón de registros que hará uso de la analítica de datos
		
\subsection{Alcance}
En el presente documento se desea mostrar ...
\subsection{Definiciones}
\textbf {UML}: Unified Modeling Language, por sus siglas en inglés, la cual traduce
Lenguaje Unificado de Modelado.\\

\textbf {HTML}: HyperText Markup Language, por sus siglas en inglés, es un lenguaje
basado en etiquetas usado en el desarrollo web el cual brinda un estándar para
la definición de la estructura y para la definición de contenido de la página web
como: texto, imágenes y videos.\\

\textbf {Angular}: Framework para desarrollo de aplicaciones web desarrollado en TypeScript, de código abierto y mantenido por Google.\\

\section{Características del producto}
En el desarrollo del proyecto se desempeñarán las buenas prácticas de administración de proyectos aprendidas hasta el momento, utilizando herramientas potentes tales como:
\begin{itemize}
\item \textit{uso del framework angular.}
\item \textit{uso de kanban automatizados.}
\item \textit{uso del modelo git branching para el control de código durante el proceso de desarrollo.}
\end{itemize}

\subsection{Objetivo}

\subsection{Características Generales}

\section{Usuarios}
\subsection{Perfil del usuario}

\section{Vistas o Diagramas}
\subsection{Diagramas de caso de uso}

\subsection{Diagrama de clases}

\subsection{Diagrama de arquitectura}
\includegraphics[scale=0.1]{uvlogo}


	 	
\section{Conclusiones} 
	 
  
\end{document}